\begin{abstract}
We have developed a flexible modular system of computer codes for
inversion of electromagnetic (EM) geophyisical data.
We first present a general mathematical framework for 
computations involving the Jacobian of the 
non-linear mapping from model parameters to observations.
Such computations arise in all gradient based inversion
methods, including variants on Gauss-Newton and non-linear conjugate
gradients.  Our analysis divides computations into components
(data functionals, forward and adjoint solvers, model parameter
mappings), and clarifies dependencies among these elements
within reallistic numerical inversion codes.  To be concrete we focus
on the 2D and 3D magnetotelluric (MT) inverse problems, but
our analysis is applicable to a wide rand of active
and passive source EM methods.  In the second part of the paper
we use this framework as a basis for development of a modular system
for EM inversion, which we have coded in Fortran95.
Inversion algorithms in the system are generic, in the sense
that they manipulate abstract data objects representing model
parameters, EM field solutions, and data vectors, without reference
to implementation details for specific problems.  As a specific
illustration of this flexibility, the same top level inversion 
modules are applied to both 2D and 3D MT problems, which are in
fact quite different.  Model paramter objects are in particular treatead 
as an abstract data type, with all object attributes private.
This construction decouples the model parameterization and
regularization from the remainder of the inversion system,
allowing greater flexibility for users, and simplifying modification 
of these critically important scientific inputs to the inversion.
The modular inversion system allows for rapid adaptation of 
inversion codes developed for one purpose 
(e.g., 3D magnetotellurics (MT)) to other EM problems
(e.g., marine controlled source EM (CSEM))
with comparatively minimal modification.
The modular approach can also simplify
maintenance of the inversion code, as
well as development of new capabilities--e.g.,
allowing for new data types such as inter-site
transfer functions in MT.
\end{abstract}
