\appendix

\subsection{3D staggered grid details}

In this section we give a slightly more precise definition
of the discrete operator corresponding to 
$\nabla \times \nabla \times - i \omega \mu \sigma$,
and clarify implentation of boundary conditions and the
meaning of adjoints in this context.
The discrete curl operator
is naturally defined as a mapping from 
all edges to all faces, but we need only consider the partial
mapping which computes the discrete curl
only for interior faces.  Denote this as
\begin{equation}
\bar \CC : \bar{\cal S}_P \mapsto {\cal S}_A
\end{equation}
and partition $\bar{\ee} \in \bar{\cal S}_P$
and $\bar\CC$ into interior and boundary edge components
\begin{equation}
\bar{\ee} = \left[ \begin{array}{c}
\ee \\ \ee_b \end{array}\right]
\,\,\,\,\,\,\,\,\,\,\,\,\,\,\,\,\,\,
\bar{\CC} = \left[ \begin{array}{c c}
\CC & \CC_b \end{array}\right]
\end{equation}
so that $\CC : {\cal S}_P \mapsto {\cal S}_A $.

To define adjoints preciscely we need to specify
inner products.  The natural inner products for the primary
and auxiliary spaces are
\begin{equation}
\label{e.IPdef}
\left< \ee_1 , \ee_2 \right>_P = \ee_1^*\VV_E\ee_2
\,\,\,\,\,\,\,\,\,\,\,
\left< \FF_1 , \FF_2 \right>_A = \FF_1^*\VV_F\FF_2 .
\end{equation}
In (\ref{e.IPdef}) $\VV_E$ ane $\VV_F$ are diagonal matrices of edge 
and face volume elements.  Edge volumes, for example, are
defined as one fourth
of the total volume of the four cells sharing the edge,
so that the first discrete inner product in (\ref{e.IPdef})
approximates the integral $L_2$ inner product for vector fields
$\int\int\int \EE_1(\ee)^*\EE_2(\xx) dV$.
The adjoint of the interior curl operator
$\CC^{\dagger} : {\cal S}_P \mapsto {\cal S}_A$
satisfies, by definition,
\begin{equation}
\left< \FF , \CC\ee \right>_A =  
\left< \CC^\dagger \FF , \ee \right>_P  
\forall \ee\in{\cal S}_P, \FF\in{\cal S}_A .
\end{equation}
Noting that $\CC$ is real, one then readily derives
\begin{equation}
\label{e.Cadjt}
\CC^\dagger = \VV_E^{-1}\CC^T\VV_F .
\end{equation}
From the definitionsof $\VV_E$ and $\VV_F$ one can verify that
$\CC^{\dagger}$ indeed corresponds to the appropriate
geometric definition of the curl operator defined on cell
faces.  Thus the electric field vector diffusion equation with
source $\jj_s$
\begin{equation}
\nabla \times \nabla \times \EE - i \omega \mu \sigma\EE
= \jj_s
\end{equation}
can be approximated on the discrete grid as
\begin{equation}
\label{e.fullDiscrete}
[\CC^{\dagger}\bar\CC -i\omega\mu\sigma]\bar\ee = 
[\CC^{\dagger}\CC -i\omega\mu\sigma]\ee + \CC^{\dagger}\CC_b\ee_b
= {\bf s} ,
\end{equation}
where $\bf s$ is the discretization of the source currents $\jj_s$,
which vanishes for the 3D MT example we have focused on.
The discrete system (\ref{e.fullDiscrete}) has one equation
for each interior node, and to complete specficiation of the
system boundary conditions are required.  We take these
as specification of the boundary electric field components,
adding
\begin{equation}
\ee_b = \ee_b^0 .
\end{equation}
Eliminating the boundary edges results in a well-posed problem
for electric fields restricted to interior edges
\begin{equation}
\label{e.Discrete}
\left[
\CC^\dagger \CC - i \omega\mu\sigma\II
\right] \ee = 
{\bf s} - \CC^\dagger\CC_b\ee_b^0 = \bb ,
\end{equation}
with the RHS determined from the boundary
data, and any source terms in the domain.

Using (\ref{e.Cadjt})  and multiplying (\ref{e.Discrete}) by
$\VV_E$ we obtain
\begin{equation}
\label{e.Symmetric}
\left[
\CC^T\VV_F\CC - i \omega\mu\sigma\VV_E
\right] \ee = \VV_E\bb ,
\end{equation}
which is clearly symetric, as noted in section 2.

\subsection{Interpolation functions for TM mode impedances}

\subsection{Degeneracy of impedance sensitivities}

\subsection{Complex Data, Real Model Parameters}

{\em [Relic text ... not deleting in case some of it might
be needed, but won't be in a paper in anything like this form!]}

The frequency domain EM differential equations
are formulated in the complex domain,
and the EM solutions are complex fields.  For simplicity
data are also generally complex, e.g., impedance components.
On the other hand the unknown model parameter,
electrical conductivity, is generally taken to be real.
As a result care is required in
interpreting the sensitivity matrix $\JJ$.  Literally
${\bf S}_{\omega, \mm}$ represents a matrix with complex entries,
so the formal expression for the Jacobian in (\ref{e.SensModel})
also represents a complex matrix which we here
call $\bar \JJ$.  The real (imaginary) part of the product
$\bar \JJ  \delta \mm$ gives the sensitivity of the real (imaginary) part of
the data, so one can separate the real and imaginary parts of $\bar \JJ$ and
the data vector to obtain a real matrix
$\JJ$ (with twice as many rows) which maps
real model parameters to real data.  This is the approach we
have implicitly taken in our discussion of inversion methods, in which
we have assumed that all matrices were real.

Using reciprocity to calculate $\bar \JJ$ with (\ref{e.SensAdjtApp})
requires solution of the adjoint equations once for each data vector
element.
The result of the formal calculation is an $M$-dimensional complex
vector, one row of $\bar \JJ$.
Thus sensitivities for real and imaginary
parts of the data are calculated with a single solution of the
transposed complex system ${\bf S}_{\omega, \mm}^T$.  Hence, if
we take $N_d$ to be the total number of real data, which actually
consist of real and imaginary parts of complex data, the number
of model solutions required for complete calculation of the Jacobian
is $N_d/2$.  Note that apparent resistivity and phase data can also
be treated essentially as real and imaginary parts of the natural
log of the complex impedance, so sensitivities for both of these
responses can also be computed with a single call to the solver.

Next consider computation of something like $\JJ^T \rr$
The actual code based on (\ref{e.SensAdjtApp})
implements multiplication of complex vectors
$\bar \rr$ by $\bar \JJ$ resulting in
a complex $M$-dimensional vector.
To compute the desired real matrix vector product
$\Re [\bar \JJ^T] \rr_r + \Im [\bar \JJ^T] \rr_i$, where
$\rr_r, \rr_i$ correspond to real and imaginary data components,
we must set $\bar \rr = \rr_r - i \rr_i$,
and take the real part of the complex product $\Re [\bar \JJ \bar \rr]$.
Key to our development of the new inversion algorithm is our
observation that $\JJ^T \rr$ is the sensitivity for a particular
linear data contrast, $\rr^T\dd$.  In the process of computing
this we obtain also $\Im [\bar \JJ \bar \rr]$. In fact, this is
the sensitivity for the data contrast  $\tilde \rr^T\dd$,
where $\tilde \rr$ is the real data vector corresponding to
$i\bar \rr$. Thus we actually generate sensitivities for
a two dimensional subspace of the real data space when
we compute $\bar \JJ^T\bar \rr$.
The additional sensitivity could in principal be used to
improve the approximation of the Jacobian, at the cost of saving
$\Im [\bar \JJ \bar \rr]$.
